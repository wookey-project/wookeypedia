\documentclass{article}
\usepackage[utf8]{inputenc}
\usepackage{tikz}
\usepackage{wrapfig}
\usepackage{tikz-qtree}
\usetikzlibrary{arrows,positioning}
\usetikzlibrary{patterns}
\usetikzlibrary{shapes}
\usetikzlibrary{automata}
\usetikzlibrary{calc}
\usetikzlibrary{fit,calc,positioning,decorations.pathreplacing,matrix,chains}
\usetikzlibrary{decorations.pathmorphing}
\usetikzlibrary{decorations.text}
\usepackage{mathtools}
\usepackage{tikz-cd}
\usepackage{verbatim}
\usepackage{color}
\usepackage{array}
\usepackage[active,tightpage]{preview}
\PreviewEnvironment{center}

\begin{document}

\begin{center}
\def\XLR#1{\xleftrightarrow[\rule{3cm}{0pt}]{#1}}% \rule defines the width
\def\XR#1{\xrightarrow[\rule{3cm}{0pt}]{#1}}
\def\XL#1{\xleftarrow[\rule{3cm}{0pt}]{#1}}

\newcommand\XRSEC[2]{%
    \mathrel{%
        \begin{tikzpicture}[%
            baseline={(current bounding box.south)}
            ]
        \node[%
            ,inner sep=.44ex
            ,align=center, minimum width=#2,
            ] (tmp) {$\scriptstyle #1$};
        \path[%
            ,draw,<-, solid
            ,decorate,decoration={%
                ,zigzag
                ,amplitude=.8pt
                ,segment length=1.2mm,pre length=3.5pt
                }
            ]
        (tmp.south east) -- (tmp.south west);
        \end{tikzpicture}
        }
    }
\newcommand\XLSEC[2]{%
    \mathrel{%
        \begin{tikzpicture}[%
            baseline={(current bounding box.south)}
            ]
        \node[%
            ,inner sep=.44ex
            ,align=center, minimum width=#2,
            ] (tmp) {$\scriptstyle #1$};
        \path[%
            ,draw,<-, solid
            ,decorate,decoration={%
                ,zigzag
                ,amplitude=.8pt
                ,segment length=1.2mm,pre length=3.5pt
                }
            ]
        (tmp.south west) -- (tmp.south east);
        \end{tikzpicture}
        }
    }
%\def\XRSEC#1{\xrightarrow[\rule{3cm}{0pt}]{#1}}
\begin{tikzpicture}[overlay, remember picture]
\node[minimum width=8.5cm, minimum height=1.74cm, fill=gray!20, yshift=1.20cm, xshift=4.4cm] (a) {};
\node[rotate=-90, yshift=.2cm] (alabel) at (a.east) {$\scriptstyle\text{\underline{Stage 1}}$};

\node[minimum width=8.5cm, minimum height=2.1cm, fill=gray!30, yshift=-1.2cm, xshift=4.4cm] (b) {};
\node[rotate=-90, yshift=.2cm] (blabel) at (b.east) {$\scriptstyle\text{\underline{Stage 2}}$};
\end{tikzpicture}
\newcommand\scalemath[2]{\scalebox{#1}{\mbox{\ensuremath{\displaystyle #2}}}}
%\resizebox{\linewidth}{!}{
\scalemath{.8}{
%\[
\begin{array}{@{} c c c @{}}
\text{\underline{Platform}}  &  & \text{\underline{Token}} \\
   \text{\underline{User} enters PetPIN}& \XR{\text{DK=PBKDF2(PetPIN)}} \\
   &  & \text{KPK=Dec}_{\text{DK}}\text{(ELK)} \\
   & \XL{\text{KPK}} &    \\
   \text{PK=Dec}_{\text{KPK}}\text{(EPK)} & &    \\
   & \XLR{\text{Establish secure channel}} &  \\
   & \XRSEC{\text{PetPIN}}{3.3cm} &  \text{PetPIN OK?}\\
   \text{\underline{User} checks PetName} & \XLSEC{\text{PetName}}{3.3cm} & \text{PetName} \\
   \text{\underline{User} enters UserPIN} & \XRSEC{\text{UserPIN}}{3.3cm} &\text{UserPIN OK?} \\
   & &  \\
    & \XLSEC{\text{Secrets}}{3.3cm} &\text{Send sensitive data} \\
%\text{[...]}
 \end{array}
%\]
}
\vskip.2cm
\scalebox{.7}{
\hskip 2.5cm
\begin{tikzpicture}
\node[draw=none, fill=gray!5, draw=black, dashed] {
\begin{tikzpicture}
\node[anchor=north west] (one) {DK: derived key from PetPIN};
\node[anchor=north west] (onebis) at (one.south west) {ELK: encrypted local key (stored in the token)};
\node[anchor=north west] (two) at (onebis.south west){KPK: platform keys encryption key};
\node[anchor=north west] (three) at (two.south west) {EPK: encrypted platform keys};
\node[anchor=north west] (gout) at (three.south west) {PK: (decrypted) platform keys};
\node[anchor=north west, xshift=.7cm] (xr) at (gout.south west) {$\xrightarrow[\rule{3.35cm}{0pt}]{\text{Send in clear}}$};
\node[anchor=north west, yshift=.3cm] (xrsec) at (xr.south west) {$\XRSEC{\text{Send inside secure channel}}{3cm}$};
\end{tikzpicture}
};
\end{tikzpicture}
}
\end{center}

\end{document}
